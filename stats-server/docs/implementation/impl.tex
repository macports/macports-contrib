\documentclass[10pt]{article}
\usepackage{fullpage}


\title{MacPorts Statistics - Google Summer of Code 2011}
\author{ Derek Ingrouville - derek@macports.org }

\date{\today}

\begin{document}

\maketitle

\setlength{\parskip}{0.3cm}

\section{Client Side - MacPorts Base}

\subsection{Install}

In order to automatically submit data at regular intervals some small changes had to be made to the installation process. These changes include installing a script which handles data submissions \texttt{submitstats.sh}, configuring \texttt{launchd} to regularly run \texttt{submitstats.sh}, and generating a unique identifier for the user submitting data.

\subsubsection{Makefile.in}
\begin{itemize}
  \item Install \texttt{submitstats.sh} to \texttt{\$(DESTDIR)\${datadir}/macports/}
  \item Run \texttt{setupstats.sh}
\end{itemize}

\subsubsection{configure.ac}

Generate a universally unique identifier to identify this MacPorts installation. The UUID is generated by \texttt{uuidgen} and stored in the variable \texttt{STATS\_UUID}

\subsubsection{Scripts}
\begin{itemize}
\item \texttt{setupstats.sh}

This script is responsible for generating and installing the file \texttt{org.stats.macports.plist}. This plist is used by \texttt{launchd} to regularly run \texttt{submitstats.sh}.

The script takes two arguments
\begin{enumerate}
  \item The path to the script that \texttt{launchd} should execute
  \item The path to the MacPorts configuration file \texttt{macports.conf}
\end{enumerate}

It will execute the script once a week. The day of the week, hour and minute are determined as follows: \newline


\textbf{Weekday:} The day of the week is determined by the machine's hardware UUID modulo 7. This is to help ensure that submissions are roughly evenly distributed throughout the week. \newline
\textbf{Hour: } The hour that \texttt{submitstats.sh} was executed. \newline
\textbf{Minute:} The minute that \texttt{submitstats.sh} was executed. \newline

The plist is installed to \texttt{/Library/LaunchAgents/org.macports.stats.plist} and then loaded by \texttt{launchctl}

\item \texttt{submitstats.sh}

This script has two responsibilities
\begin{enumerate}
  \item Check if a user is participating.
  \item Submit data only if the user if participating
\end{enumerate}
  
It takes one parameter, the path to \texttt{macports.conf}.

To determine if a user is participating it checks if the variable \texttt{stats\_participate} is set to \texttt{yes}. If it is, then \texttt{port stats submit} is executed. If the user is not participating then the script exits.

The reason this script exists is to have a lightweight tool to check if a user is participating before running \texttt{port}. This script will be executed once a week for every user, regardless of whether or not they are participating. 

\end{itemize}

\subsection{Configuration}

Added several variables to \texttt{macports.conf.in} and appropriate descriptions to \texttt{macports.conf.5}.

\begin{itemize}
  \item \texttt{stats\_participate} 
  
  This indicates whether or not a user has chosen to opt-in and share their data. Its value is either \texttt{yes} or \texttt{no}
  
  \item \texttt{stats\_url}
  
  This is the url where data should be submitted.
  
  \item \texttt{stats\_id}
  
  This is the UUID used for submissions. It is initially set to value of the autoconf variable \texttt{@STATS\_UUID@}.
\end{itemize}

\subsection{Changes to macports1.0/macports.tcl}
\begin{itemize}
  \item New Globals
  
    Added globals \texttt{stats\_participate}, \texttt{stats\_url}, \texttt{stats\_id} that correspond to configuration options. \newline
    Added deferred global \texttt{gccversion}
  \item \texttt{gcc} version check
  
    Added proc setgccinfo that is called the first time \texttt{gccversion} is read.
\end{itemize}

\subsection{Changes to pextlib1.0/curl.c - CurlPostCmd()}

Added CurlPostCmd function. This takes two Tcl parameters, the post data and the url.

Example usage is 
\begin{verbatim}
  curl post "project=macports" $url
\end{verbatim}

\subsection{The \texttt{port stats} action}
\texttt{port stats} gathers lists of all active and inactive ports as well as relevant system information. It no subaction is given \texttt{port stats} prints the system information to \texttt{stdout}.

If the \texttt{submit} subaction is given then it will encode all the collected data as a \texttt{JSON} object. It then submits this via HTTP POST to a server specified in \texttt{macports.conf}.

\texttt{JSON} encoding is done though sub-procedures contained inside the procedure for the \texttt{port stats} action.

\subsection{Changes to port/port-help.tcl}
TODO

\section{Data Format}

Transmitted data is encoded as a JSON object with four fields.

\begin{verbatim}
  {
      "id": "...",
      "os": {
          ...
      },
      "active_ports": [
          {...}, 
          ...
          {...}
      ],
      "inactive_ports": [
          {...}, 
          ...
          {...}
      ]
  }
\end{verbatim}

\begin{enumerate}
  \item id
  
  This is a string containing the user's UUID.
  
  \item os
  
  This is a JSON object containing information about the user's system.
  
  \begin{verbatim}
    "os": {
        "macports_version": "1.9.99",
        "osx_version": "10.6",
        "os_arch": "i386",
        "os_platform": "darwin",
        "build_arch": "x86_64",
        "gcc_version": "4.2.1",
        "xcode_version": "4.0"
    }
  \end{verbatim}
  
  \item active\_ports
  
  This is an array of json objects. Each object represents a single port.
  
  \begin{verbatim}
    "active_ports": [
        {
            "name": "aalib",
            "version": "1.4rc5_4"
        },
        {
            "variants": "nonls +",
            "name": "aspell",
            "version": "0.60.6_4"
        }
    ]
  \end{verbatim}
  
  \item inactive\_ports
  
  This is the same as active\_ports except that port objects represent installed inactive ports.
\end{enumerate}

\section{Server Side - Ruby on Rails}
TODO




\end{document}